% Options for packages loaded elsewhere
\PassOptionsToPackage{unicode}{hyperref}
\PassOptionsToPackage{hyphens}{url}
%
\documentclass[
]{article}
\usepackage{amsmath,amssymb}
\usepackage{lmodern}
\usepackage{ifxetex,ifluatex}
\ifnum 0\ifxetex 1\fi\ifluatex 1\fi=0 % if pdftex
  \usepackage[T1]{fontenc}
  \usepackage[utf8]{inputenc}
  \usepackage{textcomp} % provide euro and other symbols
\else % if luatex or xetex
  \usepackage{unicode-math}
  \defaultfontfeatures{Scale=MatchLowercase}
  \defaultfontfeatures[\rmfamily]{Ligatures=TeX,Scale=1}
\fi
% Use upquote if available, for straight quotes in verbatim environments
\IfFileExists{upquote.sty}{\usepackage{upquote}}{}
\IfFileExists{microtype.sty}{% use microtype if available
  \usepackage[]{microtype}
  \UseMicrotypeSet[protrusion]{basicmath} % disable protrusion for tt fonts
}{}
\makeatletter
\@ifundefined{KOMAClassName}{% if non-KOMA class
  \IfFileExists{parskip.sty}{%
    \usepackage{parskip}
  }{% else
    \setlength{\parindent}{0pt}
    \setlength{\parskip}{6pt plus 2pt minus 1pt}}
}{% if KOMA class
  \KOMAoptions{parskip=half}}
\makeatother
\usepackage{xcolor}
\IfFileExists{xurl.sty}{\usepackage{xurl}}{} % add URL line breaks if available
\IfFileExists{bookmark.sty}{\usepackage{bookmark}}{\usepackage{hyperref}}
\hypersetup{
  pdftitle={A Super Brief Tutorial of R, Rstudio, R markdown, and Python},
  pdfauthor={Me},
  hidelinks,
  pdfcreator={LaTeX via pandoc}}
\urlstyle{same} % disable monospaced font for URLs
\usepackage[margin=1in]{geometry}
\usepackage{color}
\usepackage{fancyvrb}
\newcommand{\VerbBar}{|}
\newcommand{\VERB}{\Verb[commandchars=\\\{\}]}
\DefineVerbatimEnvironment{Highlighting}{Verbatim}{commandchars=\\\{\}}
% Add ',fontsize=\small' for more characters per line
\usepackage{framed}
\definecolor{shadecolor}{RGB}{248,248,248}
\newenvironment{Shaded}{\begin{snugshade}}{\end{snugshade}}
\newcommand{\AlertTok}[1]{\textcolor[rgb]{0.94,0.16,0.16}{#1}}
\newcommand{\AnnotationTok}[1]{\textcolor[rgb]{0.56,0.35,0.01}{\textbf{\textit{#1}}}}
\newcommand{\AttributeTok}[1]{\textcolor[rgb]{0.77,0.63,0.00}{#1}}
\newcommand{\BaseNTok}[1]{\textcolor[rgb]{0.00,0.00,0.81}{#1}}
\newcommand{\BuiltInTok}[1]{#1}
\newcommand{\CharTok}[1]{\textcolor[rgb]{0.31,0.60,0.02}{#1}}
\newcommand{\CommentTok}[1]{\textcolor[rgb]{0.56,0.35,0.01}{\textit{#1}}}
\newcommand{\CommentVarTok}[1]{\textcolor[rgb]{0.56,0.35,0.01}{\textbf{\textit{#1}}}}
\newcommand{\ConstantTok}[1]{\textcolor[rgb]{0.00,0.00,0.00}{#1}}
\newcommand{\ControlFlowTok}[1]{\textcolor[rgb]{0.13,0.29,0.53}{\textbf{#1}}}
\newcommand{\DataTypeTok}[1]{\textcolor[rgb]{0.13,0.29,0.53}{#1}}
\newcommand{\DecValTok}[1]{\textcolor[rgb]{0.00,0.00,0.81}{#1}}
\newcommand{\DocumentationTok}[1]{\textcolor[rgb]{0.56,0.35,0.01}{\textbf{\textit{#1}}}}
\newcommand{\ErrorTok}[1]{\textcolor[rgb]{0.64,0.00,0.00}{\textbf{#1}}}
\newcommand{\ExtensionTok}[1]{#1}
\newcommand{\FloatTok}[1]{\textcolor[rgb]{0.00,0.00,0.81}{#1}}
\newcommand{\FunctionTok}[1]{\textcolor[rgb]{0.00,0.00,0.00}{#1}}
\newcommand{\ImportTok}[1]{#1}
\newcommand{\InformationTok}[1]{\textcolor[rgb]{0.56,0.35,0.01}{\textbf{\textit{#1}}}}
\newcommand{\KeywordTok}[1]{\textcolor[rgb]{0.13,0.29,0.53}{\textbf{#1}}}
\newcommand{\NormalTok}[1]{#1}
\newcommand{\OperatorTok}[1]{\textcolor[rgb]{0.81,0.36,0.00}{\textbf{#1}}}
\newcommand{\OtherTok}[1]{\textcolor[rgb]{0.56,0.35,0.01}{#1}}
\newcommand{\PreprocessorTok}[1]{\textcolor[rgb]{0.56,0.35,0.01}{\textit{#1}}}
\newcommand{\RegionMarkerTok}[1]{#1}
\newcommand{\SpecialCharTok}[1]{\textcolor[rgb]{0.00,0.00,0.00}{#1}}
\newcommand{\SpecialStringTok}[1]{\textcolor[rgb]{0.31,0.60,0.02}{#1}}
\newcommand{\StringTok}[1]{\textcolor[rgb]{0.31,0.60,0.02}{#1}}
\newcommand{\VariableTok}[1]{\textcolor[rgb]{0.00,0.00,0.00}{#1}}
\newcommand{\VerbatimStringTok}[1]{\textcolor[rgb]{0.31,0.60,0.02}{#1}}
\newcommand{\WarningTok}[1]{\textcolor[rgb]{0.56,0.35,0.01}{\textbf{\textit{#1}}}}
\usepackage{graphicx}
\makeatletter
\def\maxwidth{\ifdim\Gin@nat@width>\linewidth\linewidth\else\Gin@nat@width\fi}
\def\maxheight{\ifdim\Gin@nat@height>\textheight\textheight\else\Gin@nat@height\fi}
\makeatother
% Scale images if necessary, so that they will not overflow the page
% margins by default, and it is still possible to overwrite the defaults
% using explicit options in \includegraphics[width, height, ...]{}
\setkeys{Gin}{width=\maxwidth,height=\maxheight,keepaspectratio}
% Set default figure placement to htbp
\makeatletter
\def\fps@figure{htbp}
\makeatother
\setlength{\emergencystretch}{3em} % prevent overfull lines
\providecommand{\tightlist}{%
  \setlength{\itemsep}{0pt}\setlength{\parskip}{0pt}}
\setcounter{secnumdepth}{-\maxdimen} % remove section numbering
\ifluatex
  \usepackage{selnolig}  % disable illegal ligatures
\fi

\title{A Super Brief Tutorial of R, Rstudio, R markdown, and Python}
\author{Me}
\date{}

\begin{document}
\maketitle

{
\setcounter{tocdepth}{2}
\tableofcontents
}
\hypertarget{writing-and-equations}{%
\section{Writing and equations}\label{writing-and-equations}}

Textual input is just like any other word processor. If you wish to have
specific formats, you can check out the R markdown cheat sheet that is
available in the \texttt{Help\ -\textgreater{}\ Cheat\ Sheets} menu of R
Studio.

Within an R markdown file, you can choose the type of output: HTML, PDF
or Word. If you choose the \texttt{output:\ html\_notebook}, you will
get the option of previewing the document (which I find quite useful).

R markdown is \emph{very} aware of spacing in the document. For example,
if you are trying to write a header do not put a space between the
``\#'' and the section name, it will not be recognized as a section.

\#Not recognized even though RStudio colors the text like a
header\ldots{}

The same goes for options in the YAML header at the top of the document,
lists, tables, etc. These environments all rely on proper indentation to
get the desired effects.

For writing equations, I recommend using standard LaTeX commands such as
\texttt{\$\textbackslash{}alpha\^{}2\ +\ \textbackslash{}beta\ =\ \textbackslash{}frac\{1\}\{2\}\$}
gives \(\alpha^2 + \beta = \frac{1}{2}\). If you wish to have the
equation be on its own line (instead of in-line) use
\texttt{\$\$\textbackslash{}alpha\^{}2\ +\ \textbackslash{}beta\ =\ \textbackslash{}frac\{1\}\{2\}\$\$}
which gives \[\alpha^2 + \beta = \frac{1}{2}.\]

Lastly, you can do in-line R evaluation if you want. You do this by
typing something like \(e^5 =\) 148.4131591. This comes in handy if you
want to pull values from stored objects into your writing.

Again, if you need help with something in R markdown, I highly recommend
looking at the cheat sheet; it will help you do most tasks.

\hypertarget{basic-r-input}{%
\section{Basic R input}\label{basic-r-input}}

The following shows just some basic evaluation of some R code in an R
markdown file. The evaluation environment begins and ends with 3
backward ticks ```''. After the first 3 backward ticks, you have
something like
\texttt{\{r\ BLOCK\ NAME,\ option1\ =\ ...,\ option\ 2\ =\ ...,\ ...\}}
Options can specify, e.g., whether you want the code to show up
(\texttt{include\ =\ FALSE\ or\ TRUE}) and many other formatting
preferences.

Here is a super basic example of doing some R evaluation:

\begin{Shaded}
\begin{Highlighting}[]
\DecValTok{1} \SpecialCharTok{+} \DecValTok{1}
\end{Highlighting}
\end{Shaded}

\begin{verbatim}
## [1] 2
\end{verbatim}

\begin{Shaded}
\begin{Highlighting}[]
\FunctionTok{sin}\NormalTok{(}\DecValTok{2}\SpecialCharTok{*}\NormalTok{pi)}
\end{Highlighting}
\end{Shaded}

\begin{verbatim}
## [1] -2.449294e-16
\end{verbatim}

\begin{Shaded}
\begin{Highlighting}[]
\FunctionTok{runif}\NormalTok{(}\DecValTok{10}\NormalTok{)}
\end{Highlighting}
\end{Shaded}

\begin{verbatim}
##  [1] 0.0278881 0.2014704 0.8400076 0.7582733 0.4261774 0.7759869 0.5935753
##  [8] 0.4359808 0.5137333 0.9060723
\end{verbatim}

\begin{Shaded}
\begin{Highlighting}[]
\NormalTok{x }\OtherTok{\textless{}{-}} \FunctionTok{rnorm}\NormalTok{(}\DecValTok{100}\NormalTok{)}
\NormalTok{y }\OtherTok{\textless{}{-}} \DecValTok{3}\SpecialCharTok{*}\NormalTok{x }\SpecialCharTok{+} \DecValTok{5} \SpecialCharTok{+} \FunctionTok{runif}\NormalTok{(}\DecValTok{100}\NormalTok{, }\SpecialCharTok{{-}}\DecValTok{1}\NormalTok{, }\DecValTok{1}\NormalTok{)}
\FunctionTok{plot}\NormalTok{(x,y)}
\end{Highlighting}
\end{Shaded}

\includegraphics{R_Tutorial_files/figure-latex/baby block-1.pdf}

\hypertarget{python-in-r-markdown}{%
\section{Python in R markdown}\label{python-in-r-markdown}}

Using Python in R requires the installation of the \texttt{reticulate}
library.

Once that package is installed and loaded, you can make python calls by
specifying \texttt{\{python\ ...\}} after the intial 3

\begin{Shaded}
\begin{Highlighting}[]
\ImportTok{import}\NormalTok{ matplotlib.pyplot }\ImportTok{as}\NormalTok{ plt}
\ImportTok{import}\NormalTok{ numpy }\ImportTok{as}\NormalTok{ np}
\end{Highlighting}
\end{Shaded}

\hypertarget{including-plots-sub-header-example}{%
\subsection{Including Plots (sub-header
example)}\label{including-plots-sub-header-example}}

Here I continue using the Python interpreter to create a plot:

\begin{Shaded}
\begin{Highlighting}[]
\NormalTok{t }\OperatorTok{=}\NormalTok{ np.arange(}\FloatTok{0.0}\NormalTok{, }\FloatTok{2.0}\NormalTok{, }\FloatTok{0.01}\NormalTok{)}
\NormalTok{s }\OperatorTok{=} \DecValTok{1} \OperatorTok{+}\NormalTok{ np.sin(}\DecValTok{2}\OperatorTok{*}\NormalTok{np.pi}\OperatorTok{*}\NormalTok{t)}

\NormalTok{plt.plot(t, s)}
\end{Highlighting}
\end{Shaded}

\begin{verbatim}
## [<matplotlib.lines.Line2D object at 0x7fecd7563668>]
\end{verbatim}

\begin{Shaded}
\begin{Highlighting}[]
\NormalTok{plt.xlabel(}\StringTok{\textquotesingle{}time (s)\textquotesingle{}}\NormalTok{)}
\end{Highlighting}
\end{Shaded}

\begin{verbatim}
## Text(0.5, 0, 'time (s)')
\end{verbatim}

\begin{Shaded}
\begin{Highlighting}[]
\NormalTok{plt.ylabel(}\StringTok{\textquotesingle{}voltage (mv)\textquotesingle{}}\NormalTok{)}
\end{Highlighting}
\end{Shaded}

\begin{verbatim}
## Text(0, 0.5, 'voltage (mv)')
\end{verbatim}

\begin{Shaded}
\begin{Highlighting}[]
\NormalTok{plt.grid(}\VariableTok{True}\NormalTok{)}
\CommentTok{\#plt.savefig("test.png")}
\NormalTok{plt.show()}
\end{Highlighting}
\end{Shaded}

\includegraphics{R_Tutorial_files/figure-latex/load libraries-1.pdf}

\end{document}
